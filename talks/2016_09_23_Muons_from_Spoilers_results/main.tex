\documentclass[xcolor={dvipsnames}]{beamer}
\usepackage{color, colortbl}
\usepackage{transparent}
\usepackage[ngerman,english]{babel}
\usepackage[T1]{fontenc}
\usepackage{lmodern}
%\usepackage{subfigure}
%\usepackage[compatibility=false]{caption}
%\usepackage{subcaption}
\usepackage{tikz}
\usepackage{textgreek}
\usepackage{tabularx}
\usepackage{booktabs}
\usepackage{siunitx}
\usepackage{units}
\usepackage[absolute,overlay]{textpos} %for positioning the logos where I want

\mode<presentation>
{
  \usetheme{CambridgeUS}     
  \usecolortheme{lily} 
  \definecolor{beamer@violet}{rgb}{0.5,0.3,0.5} % changed this
  \setbeamercolor{structure}{fg=beamer@violet!70!cyan}
  \setbeamercolor{palette primary}{fg=black, bg=gray!30!white!50!cyan!20!}
  \setbeamercolor{palette secondary}{fg=black, bg=gray!30!white!30!cyan!40!}
  \setbeamercolor*{palette tertiary}{bg=gray!20!white!20!cyan!60!}
  
  \setbeamercolor{frametitle}{fg=cyan!60!white!40!,bg=cyan!80!black}
  \setbeamercolor{title}{fg=cyan!80!black}
  \setbeamercolor{normal text}{fg=black,bg=white}
  \setbeamercolor{alerted text}{fg=beamer@violet}
  \setbeamercolor{example text}{fg=beamer@violet!70!cyan}
  
  \usefonttheme{structureitalicserif} 
  \setbeamertemplate{navigation symbols}{}
  \setbeamertemplate{caption}[numbered]
} 

\newcommand{\sidlogo}{
  \setlength{\TPHorizModule}{1pt}
  \setlength{\TPVertModule}{1pt}
   % textblock{}{x,y}: pos(x) = rightUpperCorner + (x * \TPHorizModule), pos(y) = leftUpperCorner - (y * \TPVertModule)
  \begin{textblock}{1}(323,12)
   \includegraphics[width=40pt,height=26pt]{figures/SiD.jpeg}
  \end{textblock}
  } 
\newcommand{\ilclogo}{
  \setlength{\TPHorizModule}{1pt}
  \setlength{\TPVertModule}{1pt}
   % textblock{}{x,y}: pos(x) = rightUpperCorner + (x * \TPHorizModule), pos(y) = leftUpperCorner - (y * \TPVertModule)
  \begin{textblock}{1}(323,12)
   \includegraphics[width=40pt,height=26pt]{figures/ILC.jpeg}
  \end{textblock}
} 

\title[ILC \& Muons from spoilers]{\textbf{\LARGE The International Linear Collider \\ \small Muons from the muon spoilers in the SiD detector - \\FIRST RESULTS}}
\author{\textbf{Anne Sch\"utz}}
\institute{\textbf{DESY}}
\date{\textbf{23rd September 2016}}

\titlegraphic{\includegraphics[height=1.0cm]{figures/SiD.jpeg}\hspace*{6cm}~%
   \includegraphics[height=1.0cm]{figures/DESY_Logo.png}
}

\begin{document}
{
\usebackgroundtemplate{
 \tikz\node[opacity=0.1]{\hspace*{-.05in}\vspace*{-4in}{\transparent{0.1}\includegraphics[width=\paperwidth]{muons_eventdisplay.jpeg}}};
 % \tikz\node[opacity=0.2]{\centering\includegraphics[height=\paperheight]{Iwatecomics.jpg}};
 }
\begin{frame}
  \titlepage
\end{frame}
}
\begin{frame}
  \tableofcontents
\end{frame}

\section{Muons from the muon spoilers}
\begin{frame}{The layout of the ILC}
\ilclogo
\begin{center}
\includegraphics[width=\textwidth]{figures/ILC_schematic_layout.png}
\end{center}
The muon spoilers will be installed in the Beam Delivery System (BDS) in the central region.
\end{frame}

\begin{frame}{BDS tunnel layout}
\ilclogo
\begin{center}
\includegraphics[height=0.65\textheight]{BDS_electron_tunnel.pdf}
\end{center}
\end{frame}

\subsection{Muon spoiler scenarios}
\begin{frame}{Muon spoiler scenarios}
\ilclogo
There are two spoiler scenarios under discussion:
\begin{itemize}
 \item 3 donut spoilers
 \item 3 donut spoilers + wall
\end{itemize}
\begin{center}
\includegraphics[height=0.7\textheight]{Muon_spoiler_scenarios.pdf}
\end{center}
\end{frame}

\begin{frame}{3 donut spoilers}
\ilclogo
\textbf{The donut spoilers} are designed as follows:
\begin{itemize}
 \item \unit[70]{cm} radius
 \item \unit[5]{m} long
 \item Magnetized iron with a field of $\sim$\unit[10-19]{kG}
\end{itemize}
\begin{center}
\includegraphics[height=0.67\textheight]{Muon_spoilers.pdf}
\end{center}
\end{frame}

\begin{frame}{3 donut spoilers + wall}
\ilclogo
\textbf{The iron wall} would completely fill up the tunnel:
\begin{itemize}
 \item \unit[5]{m} x \unit[3]{m}, \unit[5]{m} long
 \item Magnetized with a field of $\sim$\unit[16]{kG}
 \item Located $\sim$\unit[400]{m} away from the IP
 \item Would cost $\sim$ \$3 million
\end{itemize}
\begin{center}
\includegraphics[height=0.7\textheight]{Muon_wall.pdf}
\end{center}
\end{frame}

\section{MUCARLO simulation}
\begin{frame}{MUCARLO simulation overview}
\ilclogo

\begin{columns}
 \begin{column}{0.7\textwidth}
  \begin{itemize}
\item BDS backgrounds with muon collimation system modelled with MUCARLO [Lewis Keller, SLAC] and Geant4 [Glen White, SLAC]
\item Using TDR baseline machine parameters for the ILC500
\item Muon production processes:
\begin{itemize}
\item Predominantly: Bethe-Heitler process:\\ \textgamma + Z $\rightarrow$ Z' + \textmu$^+$\textmu$^-$
\item Few \% level: direct annihilation of positrons with atomic electrons: e$^+$e$^-$ $\rightarrow$ \textmu$^+$\textmu$^-$
\end{itemize}
\item Halo particle tracking:
\begin{itemize}
\item Turtle with MUCARLO
\item Lucretia with a built-in Geant4 model interface
\end{itemize}
\end{itemize}

 \end{column}
 \begin{column}{0.3\textwidth}
  \includegraphics[width=\textwidth]{BetheHeitler.pdf}
 \end{column}
\end{columns}
\end{frame}

\subsection{Muon tracking}
\begin{frame}{Muon tracks in the BDS tunnel}
\ilclogo
Muon tracks of positively (\textcolor{green}{\textmu\textsuperscript{+}}) and negatively (\textcolor{red}{\textmu\textsuperscript{-}}) charged muons, originating at two different primary betatron collimators: SP2 and SP4.
\begin{center}
\includegraphics[width=0.95\textwidth]{Muon_tracks.pdf}
\end{center}
The tracks that are drawn are only the ones that reach the detector.\\
Because the SP2 is further away and the negative muons are deflected by the magnetized spoilers, the negative muons are not drawn because they don't reach the detector.
\end{frame}

\subsection{Muon 4-vectors}
\begin{frame}{Muon 4-vectors}
\ilclogo
\begin{center}
\includegraphics[height=0.5\textheight]{Muon_4-vectors.pdf}
\end{center}
Spatial and momentum distribution of muons in a detector (of 6.5m radius).
4-vectors provided to SiD and ILD.
\end{frame}
\begin{frame}{Muons in the detector}
\ilclogo
\begin{center}
\includegraphics[height=0.6\textheight]{Muon_numbers.pdf}
\end{center}
First column: muon numbers per bunch in a detector with 6.5m radius.
The number can be reduced to below 1!
\end{frame}

\begin{frame}{Attenuation Factor}
\ilclogo
\begin{center}
\includegraphics[height=0.7\textheight]{Attenuation_Factors.pdf}
\end{center}
The ratio of muons produced over muons which reach the detector, for different spoiler conditions and different source locations.
\end{frame}


\section{Motivation}
\begin{frame}{}
\ilclogo
Question to SiD and ILD: Do we need the muon wall at all?!
MID people would be happy to get rid of it because of safety issues.
\begin{center}
\includegraphics[height=0.5\textheight]{Muon_wall_required.pdf}
\end{center}
\end{frame}

\section{First results}
\begin{frame}{Summer student project}
\ilclogo
The first analysis of the muon hits and the detector occupancy in the SiD detector was done by Jonas Glombitza, Marcel's and my summer student this summer 2016.\\
\vspace*{0.5cm}
Preparations:
\begin{itemize}
\item 4-vector files from Lewis Keller:
\begin{itemize}
 \item Spoilers + wall: from electron line: $\sim$1500 muons
 \item Spoilers + wall: from positron line: $\sim$2100 muons
 \item Spoilers: from electron line: $\sim$1080 muons
 \item Spoilers: from positron line: $\sim$2280 muons
\end{itemize}
\item Conversion of the text files with the 4-vector values to  STDHEP files of 1 train worth of muons.
\item The STDHEP files were used as input to a full SiD detector simulation with SLIC.
\item Nice event displays from the simulations with WIRED4 in JAS3.
\end{itemize}

\end{frame}

\subsection{Event displays of muons in the SiD detector}
\begin{frame}{WIRED4 event display}
\sidlogo
1 train's worth of muons ($\sim$ 650 muons):
\begin{center}
\includegraphics[height=0.6\textheight]{sidloi3_muons_wired4_eventdisplay_1bunch.png}
\hspace*{0.2cm}
\includegraphics[height=0.6\textheight]{sidloi3_muons_wired4_eventdisplay_xy_view_1bunch.png}
\end{center}
The asymmetry in the xy plane is predicted by the MUCARLO simulation output (see a few slides before), and clearly visible also in the SLIC simulation.
\end{frame}

%New command: column type
\newcolumntype{P}[1]{>{\centering}p{#1}}

\subsection{Analysis - Energy distributions}
\begin{frame}{Energy distribution of muons -\\ \small Spoiler and Spoiler+Wall scenarios}
\sidlogo
 \begin{center}
\includegraphics[height=0.6\textheight]{Energy_distributions.pdf}
\end{center}
\small This is showing equal number of events. The real absolute number of muons cannot be analyzed yet because more simulation files from Lewis Keller are missing. There would be a lot more events for the Spoiler-only scenario.
\end{frame}

\subsection{Analysis - Spatial distributions}
\begin{frame}{Spatial distribution in the MuonEndcaps - \\ \small Spoiler and Spoiler+Wall scenarios}
\sidlogo
Hits from muons from 5 trains for both MuonEndcaps and all their layers:
 \begin{center}
\includegraphics[width=0.5\textwidth]{Spatial_distribution_MuonEndcap_Spoiler.pdf}
\hspace*{0.1cm}
\includegraphics[width=0.5\textwidth]{Spatial_distribution_MuonEndcap_SpoilerWall.pdf}
\end{center}
\end{frame}
\begin{frame}{Angular distributions of muons \small before hitting the SiD}
\sidlogo
Muons from 5 trains:
 \begin{center}
\includegraphics[width=0.5\textwidth]{horizontalAngular_distributions.pdf}
\hspace*{0.1cm}
\includegraphics[width=0.5\textwidth]{verticalAngular_distributions.pdf}
\end{center}
\end{frame}
\begin{frame}{Explanation of spatial distributions \small in the MuonEndcaps}
\sidlogo
 \begin{center}
\includegraphics[height=0.85\textheight]{Explanation_Spatial_distribution.pdf}
\end{center}
\end{frame}

\subsection{Analysis - Total number of hits}
\begin{frame}{Total number of hits - \small Spoiler scenario}
\sidlogo
 \begin{center}
\includegraphics[height=0.65\textheight]{Number_Hits_per_Subdetector.pdf}\\
\begin{tabular}{@{}p{0.343\textwidth}p{0.01\textwidth}p{0.18\textwidth}p{0.01\textwidth}p{0.343\textwidth}p{0.001\textwidth}@{}}
 \centering Vertex detectors & < & \centering ECAL, HCAL & < & \centering MuonEndcaps & \\
  \centering{\scriptsize Smallest effective detector area} & &  \centering{\scriptsize Particle showers} & &  \centering{\scriptsize Biggest effective detector area}&
\end{tabular}
\end{center}
\end{frame}
\begin{frame}{Explanation of hit number distribution -\\ \small Spatial distribution in the MuonEndcaps}
\sidlogo
 \begin{center}
\includegraphics[height=0.78\textheight]{Explanation_Hits_Subdetectors.pdf}
\end{center}
\end{frame}
\begin{frame}{Total number of hits - \small Spoiler+Wall scenario}
\sidlogo
 \begin{center}
\includegraphics[height=0.65\textheight]{Number_Hits_per_Subdetector_SpoilerWall.pdf}\\
Number of hits more equally distributed, but still:\\
\begin{tabular}{@{}p{0.343\textwidth}p{0.01\textwidth}p{0.18\textwidth}p{0.01\textwidth}p{0.343\textwidth}p{0.001\textwidth}@{}}
 \centering Vertex detectors & < & \centering ECAL, HCAL & < & \centering MuonEndcaps & \\
  \centering{\scriptsize Smallest effective detector area} & &  \centering{\scriptsize Particle showers} & &  \centering{\scriptsize Biggest effective detector area}&
\end{tabular}
\end{center}
\end{frame}

\subsection{Analysis - Occupancies}
\begin{frame}{Occupancy plots - \small MuonEndcap}
\sidlogo
 \begin{center}
\includegraphics[height=0.7\textheight]{Occupancy_MuonEndcap.pdf}
\end{center}
The muon background has a small impact on the MuonEndcaps.
\end{frame}
\begin{frame}{Occupancy plots - \small SiTrackerBarrel}
\sidlogo
 \begin{center}
\includegraphics[height=0.68\textheight]{Occupancy_SiTrackerBarrel.pdf}
\end{center}
\footnotesize 10$^{-9}$ - 10$^{-8}$ of all cells that get hit, have 9 hits.\\
\small Spoiler+Wall seems to do better by a factor of 3-5, with the same initial number of events. BUT final statement cannot be made yet!
\end{frame}

\subsection{Analysis - Time distributions}
\begin{frame}{Time distribution - \small SiTrackerEndcaps}
\sidlogo
Note that the timing of the muons is not 100\% accurate yet! Lewis Keller will send the accurate numbers soon.
 \begin{center}
\includegraphics[height=0.78\textheight]{Time_distribution_SiTrackerEndcaps.pdf}
\end{center}
\end{frame}
\begin{frame}{Time distribution - \small MuonEndcaps}
\sidlogo
 \begin{center}
\includegraphics[height=0.78\textheight]{Time_distribution_MuonEndcaps.pdf}
\end{center}
\end{frame}

\section{Conclusion and Outlook}
\begin{frame}
\textit{Conclusion:}
\begin{itemize}
\item Low energy muons are stopped by the muon wall.
\item High energy muons could be used for tracker alignment.
\item Spatial distributions quite different in the Spoiler and Spoiler+Wall scenarios.
\item Number of hits in subdetectors are explained by geometries.
\item Occupancy is small, but final statement cannot be made yet.
\item Muons are instantaneous in comparison to pair background.
\end{itemize}
\textit{Outlook:}
\begin{itemize}
\item Need to wait for more files from Lewis. He is cross-checking his results with Glen White at the moment.
\item The timing information need to be updated.
\item Finalizing statement about absolute numbers of muons in both scenarios (spoilers, spoilers + wall), and about detector occupancies
\end{itemize}
\alert{Maybe a final conclusion about whether a muon wall is needed will have to wait till then...\\
$\rightarrow$\textit{Stay tuned!}}
\end{frame}

\section{References}
\begin{frame}{References}
\tiny
\begin{thebibliography}{9}
\setbeamertemplate{bibliography item}[text]
\bibitem{MUCARLO_talk}  \emph{ECFA 2016: Talk by Glen White about the MUCARLO simulation of the muons from the muon spoilers}. \url{https://agenda.linearcollider.org/event/7014/contributions/34689/attachments/30076/44961/ILC_muons.pptx}
\bibitem{Jonas_talk}  \emph{DESY summer student program: Talk by Jonas Glomitza (RWTH Aachen) about ``The Impacts of the Muon Spoiler Background on the ILC Detector Performance'', 08. September 2016}. \url{https://indico.desy.de/getFile.py/access?contribId=9&resId=0&materialId=slides&confId=15972}
\bibitem{Suppression}  \emph{FERMILAB-CONF-07-276-AD: ``Suppression of Muon Backgrounds generated in the ILC Beam Delivery System'', Drozhdin et.al, 2007}. \url{https://inspirehep.net/record/771808/files/fermilab-conf-07-276.pdf}
\bibitem{MUCARLO}  \emph{``Calculation of Muon Background in Electron Accelerators using the Monte Carlo Computer Program MUCARLO'', Rokni et.al}. \url{http://www.slac.stanford.edu/cgi-wrap/getdoc/slac-pub-7054.pdf}
\bibitem{MuonBackground_1TeV}  \emph{SLAC-PUB-6385: ``Muon Background in a 1.0-TeV Linear Collider'', L.P. Keller, 1993}. \url{http://www.slac.stanford.edu/pubs/slacpubs/6250/slac-pub-6385.pdf}
\bibitem{MuonBackground_0.5TeV}  \emph{SLAC-PUB-5533: ``Calculation of Muon Background in a 0.5 TeV Linear Collider'', L.P. Keller, 1991}. \url{http://www.slac.stanford.edu/cgi-wrap/getdoc/slac-pub-5533.pdf}
\end{thebibliography}
\end{frame}
%%--------------------------------------------------------------------------------
\section*{Appendix}
\begin{frame}{Attenuation Factor}
\ilclogo
\begin{center}
\includegraphics[height=0.5\textheight]{Attenuation_Factors.pdf}
\end{center}
Spoilers (SP) have a much smaller gap than the absorbers (PC and AB), and they intercept the primary beam halo.\\
Muons originating at the SPs:
\begin{itemize}
 \item smaller radius $\rightarrow$ ``see'' more of the detector
 \item smaller angular spread (SPs are at max-beta points = min-divergence)
\end{itemize} 
\end{frame}

\end{document}
