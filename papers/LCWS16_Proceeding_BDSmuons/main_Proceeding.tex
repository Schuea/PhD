\pdfoutput=1 % for arXiv to use pdflatex
\documentclass[12pt]{article}

%% The graphicx package provides the includegraphics command.
\usepackage{graphicx}
%% The amssymb package provides various useful mathematical symbols
\usepackage{amsmath}
\usepackage{amssymb}
%% The lineno packages adds line numbers. Start line numbering with
%% \begin{linenumbers}, end it with \end{linenumbers}. Or switch it on
%% for the whole article with \linenumbers after \end{frontmatter}.
\usepackage{lineno}
\usepackage{url}
\usepackage{xspace,multicol}
\usepackage{siunitx}
\usepackage{subcaption}
\usepackage{color, colortbl}
\usepackage{units}
\usepackage{ragged2e}
\usepackage{array}
\usepackage{tabularx}
\usepackage{authblk}
\usepackage{heppennames}
\usepackage{feynmp}
\usepackage{libertine}
% \usepackage[libertine]{newtxmath}
\usepackage{textgreek}
\DeclareGraphicsRule{*}{mps}{*}{}

\newcommand{\guineapig}{GuineaPig\xspace}
\newcommand{\sid}{SiD\xspace}
\newcommand{\slic}{\textsc{SLIC}\xspace}
\newcommand{\geant}{\textsc{Geant4}\xspace}
\newcommand{\murm}{%
  \ifmmode
    \mathchoice
        {\hbox{\normalsize\textmu}}
        {\hbox{\normalsize\textmu}}
        {\hbox{\scriptsize\textmu}}
        {\hbox{\tiny\textmu}}%
  \else
    \textmu
  \fi
}
\newcommand{\micron}{\ensuremath{\murm\mathrm{m}\xspace}}

\begin{document}

%% Title, authors and addresses

\title{A Study of the Impact of Muons from the Beam Delivery System on the SiD Performance}

\author{Anne Sch\"utz\footnote{Karlsruhe Institute of Technology (KIT), Department of Physics, Institute of Experimental Nuclear Physics (IEKP), Wolfgang-Gaede-Str. 1, 76131 Karlsruhe, Germany}, Marcel Stanitzki\footnote{Deutsches Elektronen-Synchrotron (DESY), Notkestr. 85, 22607 Hamburg, Germany}}

\maketitle

%%
%% Start line numbering here if you want
%%
\linenumbers

\begin{abstract}
%% Text of abstract
To suppress the muon background arising from the Beam Delivery System (BDS) of the International Linear Collider (ILC), and to hinder it from reaching the interaction region, two different shielding scenarios are under discussion: five cylindrical muon spoilers with or without an additional magnetized shielding wall.
Due to cost and safety issues, the case preferred by the MDI group is to omit the shielding wall, which on the other hand brings other disadvantages.
To support the decision making, the impact of the muons from the two different shielding scenarios was studied in a full Geant4 detector simulation of the SiD detector, one of two proposed detectors for the ILC. 
Input to this study is the muon background created by the beam traveling through the BDS, which was simulated with the simulation tool MUCARLO.
\end{abstract}


%% main text
\section{Introduction}
\label{sec:introduction}

Although much more forgiving than for LHC collisions, activity generated by ILC collisions still needs to be considered in the geometric and readout design of SiD detector components, particularly those closest to the beamline. In this study, we have considered the impact of ILC beam-beam collision activity on the pixelization (spatial and temporal) of the pixel vertex detector (VXD), the readout architecture of the electromagnetic calorimeter (ECAL), and the geometry and layout of the Beamline Calorimeter (BeamCal). We have also considered the impact of the longitudinal position of the BeamCal (tied to the value of $L^*$, the distance from the interaction point to the tip of the final focusing quadrupole), and the inclusion of a dedicated magnetic field from an anti-detector-integrated dipole (anti-DiD), intended to sweep beam-beam collision products into the ILC beam exit pipe, on the backgrounds that populate these innermost detector elements.

As currently envisioned, the ILC machine will deliver beam trains at a rate of \unit[5]{Hz}. Three sets of possible beam parameters, as defined in the Technical Design Report~\cite{TDR}, are listed in Table~\ref{tab:ILC_parameters}. At the center-of-momentum energy \mbox{$E_\text{CM} = \unit[500]{GeV}$}, in the energy region for which the ILC would be expected to run in its first years of operation, two sets of parameters are presented. For the baseline (lower luminosity) set, each train would consist of 1312 crossings each separated by \unit[554]{ns}, for a total train duration of 0.72 ms, and with an integrated luminosity of \unit[3.6]{nb$^{-1}$} per train. A higher-luminosity \unit[500]{GeV} scenario envisions a train of 2625 bunches each separated by \unit[366]{ns}, with total train duration of \unit[0.96]{ms} and an integrated luminosity if \unit[7.2]{nb$^{-1}$} per train. A third set of parameters, proposed for running at an upgraded center-of-momentum energy of \unit[1]{TeV}, was not considered in this study. In all cases, though, the interval of \unit[199]{ms} between successive trains can be made use of for digital processing of analog signal pulses. Design considerations due to high cross-section background processes arise both from the effect of individual crossings, as well as from the effect of multiple beam crossings integrated over portions of the, or the entire, beam train.

\section{The Simulation of the Muon Background with MUCARLO}
\label{MUCARLO}

The simulation code MUCARLO\cite{MuonBkg_05TeV, MuonBkg_1TeV} is based on Fortran, and was originally written by Gary Feldman.
Over the years it has been expanded, and is used in several studies, from the study of muon shielding designs for radiation protection, to fixed target experiments at SLAC and muon background simulation studies for the Next Linear Collider (NLC) and the ILC\cite{MuonBkg_05TeV, MuonBkg_1TeV}.\\
For the presented study, the Technical Design Report (TDR) baseline machine parameters for the ILC-500GeV are used for simulating the beam interacting with the BDS geometry and the muon collimation system.
The muons are produced in interactions of the beam halo with material in the beam lines, in which the predominant interaction is the Bethe-Heilter process:
\textgamma + Z $\rightarrow$ Z' + \murm\textsuperscript{+}\murm\textsuperscript{-}\\
The muon production by direct annihilation of the positrons with atomic electrons is also taken into account.\cite[sec. 2]{Mucarlo}
For tracking the beam halo, the tool TURTLE\cite{Turtle} is used.\\
The results from the MUCARLO simulations can be seen in Table~\ref{tab:MuonRates}, listing the number of muons reaching the interaction region for the two shielding scenarios and for the case of not having any muon shielding system.
The calculated muon rate is based on a halo population of 10\textsuperscript{-3}, which is more than ten times larger than expected from ring scattering calculations.
This estimation corresponds to the worst halo measured at the Stanford Linear Collider (SLC), and is therefore used as a worst-case scenario.

\begin{table}
\caption{MUCARLO results: The number of muons hitting a detector with radius of \unit{6.5}{m} in the different shielding scenarios.}
\label{tab:MuonRates}
\centering
\begin{tabularx}{\textwidth}{ll}
\hline\hline
\textbf{Scenario} & \textbf{Number of muons in a detector with 6.5m radius}\\
\hline
\cline{1-2}
\hline
 No Spoilers & 130 muons/bunch crossing\\
 5 Spoilers& 4.3 muons/bunch crossing\\
 5 Spoilers + Wall & 0.68 muons/bunch crossing\\
\hline\hline
\end{tabularx}
\end{table}
\section{The simulation of muons in the SiD detector}
\label{Detector}

In the first step of simulating the muons in the SiD detector, event displays were made with WIRED4.

\begin{figure}
    \centering
    \begin{subfigure}[b]{0.49\textwidth}
    \centering
        \includegraphics[height=0.3\textheight]{figures/muons_positron_5spoilers_wall_515_xyview_croped.png}
        \caption{xy-view}
	\label{fig:xy_5Spoilers}
    \end{subfigure}
    \begin{subfigure}[b]{0.49\textwidth}
    \centering
        \includegraphics[height=0.3\textheight]{figures/muons_positron_5spoilers_wall_515_zyview_croped.png}
        \caption{xy-view}
        \label{fig:zy_5Spoilers}
    \end{subfigure}
    \caption[Event displays of muons in SiD from the '5 Spoilers' scenario]{
    Event displays in the xy- and zy-view of the muons from the '5 Spoilers' scenario in the SiD detector.
    }
    \label{fig:WIRED4_5Spoilers}
\end{figure}

\begin{figure}
    \centering
    \begin{subfigure}[b]{0.49\textwidth}
    \centering
        \includegraphics[height=0.3\textheight]{figures/muons_positron_5spoilers_2961_xyview_croped.png}
        \caption{xy-view}
	\label{fig:xy_5SpoilersWall}
    \end{subfigure}
    \begin{subfigure}[b]{0.49\textwidth}
    \centering
        \includegraphics[height=0.3\textheight]{figures/muons_positron_5spoilers_2961_zyview_croped.png}
        \caption{xy-view}
        \label{fig:zy_5SpoilersWall}
    \end{subfigure}
    \caption[Event displays of muons in SiD from the '5 Spoilers + Wall' scenario]{
    Event displays in the xy- and zy-view of the muons from the '5 Spoilers + Wall' scenario in the SiD detector.
    }
    \label{fig:WIRED4_5SpoilersWall}
\end{figure}
\section{Summary and conclusion}
By looking at the luminosity spectra generated for this study, the expected rise in luminosity for the proposed ILC250 parameter sets is confirmed.
The total luminosity resulting from the new official beam parameter set (A) is increased by about \SI{97}{\percent} with respect to the original beam parameters, leading to a new value of \SI{1.62e34}{\per\centi\meter\squared\per\second}.
The background studies indicate that the new beam parameters increase the \Pep\Pem pair background density and the SiD vertex detector occupancy by only a factor of about 2-3 in comparison to the original TDR beam parameters.
As for a precision detector the aim is to keep the overall ratio of so-called ``dead'' cells with a full buffer below a per mill level, any rise in the occupancy level can be critical.
Although the density of the pair background close to the interaction point rises, the normalized vertex detector occupancy for the new set is well below 10$^{-4}$.
The highest occupancy is reached in the innermost layer with a fraction of dead cells of about \num{5e-5}, assuming a buffer depth of four.\\
The SiD Optimization group is confident that this rise in the occupancy can be accommodated in the design of the SiD vertex detector without a loss in precision for the physics studies.
Therefore, the SiD group welcomes the decision to change the beam parameters in order to gain higher luminosities and to strengthen the physics program.


\section*{Acknowledgments}
Grateful acknowledgments to Lewis Keller and Glen White from SLAC for the simulation of the muons from the BDS and their big support.

\include{Appendix}

%% References with bibTeX database:

\bibliographystyle{abbrv}
%\bibliographystyle{model1-num-names} %Doesn't work well if you also cite webpages (Confluence page)
\bibliography{PairBackground_in_SiD.bib}

\end{document}
