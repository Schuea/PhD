\section{Summary and Conclusions}

Making use of the SiD simulation infrastructure, we have conducted a simulation study of several aspects of the performance of the SiD vertex detector, forward electromagnetic calorimeter, and BeamCal. In particular, we have explored the way in which collision backgrounds (primarily from the production of coherent electron-positron pairs, but also including two-photon events when necessary) impact the performance of these detector elements. Within this context, we have also explored the effect of various IP design choices, including L* and BeamCal geometry variants as well as the inclusion of an anti-DiD magnetic field configuration that would sweep much of the coherent pair background into the exiting beampipe, thereby potentially improving the BeamCal reconstruction efficiency and lowering vertex detector backgrounds.

Fractional channel occupancy was used as a measure of the impact of the pair background on the performance of the vertex detector. Even under the conservative assumption of a pixel size of $\unit[30\times30]{\micron^{2}}$ and a readout that integrates over five bunch crossings, the occupancy was found to be well below $10^{-3}$ for any point in the barrel portion, and for any point in the endcap except at the innermost radii, where it rose to approximately $2 \times 10^{-3}$. Studies of the point-of-origin of pair-induced vertex detector background showed that only a fraction of the background arises from the BeamCal albedo, making the result insensitive to the L*, anti-DiD, and BeamCal geometry variants that were studied.

Furthermore, the studies of the hit time and creation time distributions of the background particles have shown that the pairs from the pair background do not hit the vertex detector instantaneously at the time of the bunch crossing. Due to the effect of backscattering and their low transverse momentum, pair background particles hit the vertex detector up to several microseconds late. These broad time distributions will allow us to reject hits that occur more than \unit[20]{ns} after the instant of the beam crossing by use of an electronic gate.

Making use of a simulation of all potentially high-occupancy backgrounds (pair, Bhabha, and two-photon production), we have explored the necessary buffer depth of the forward electromagnetic calorimeter readout. For the forward calorimeter overall, a depth of four buffers leads to a loss of several tenths of a percent of hits. Increasing this to six reduces the hit loss to just above  $10^{-4}$. However, while this result was also observed when the layer-by-layer hit loss was evaluated, the inverse relationship between background rate and distance from the beampipe led to a requirement of eight buffers to maintain hit losses below $10^{-4}$ for the innermost radii of the forward electromagnetic calorimeter.

Finally, our study of the dependence of the BeamCal reconstruction efficiency upon the choice of L* showed little effect, with a slight improvement in the geometrical acceptance observed for the larger, preferred value of L*. The inclusion of the anti-DiD field, which can have no effect on the geometrical acceptance, was observed to provide a noticeable improvement in the inefficiency for tagging \unit[50]{GeV} electrons in the region of the BeamCal between 30 and \unit[40]{cm} from the exiting electron/positron beams.

%[Final flowery statements; acknowledgements?]

%\subsection{Electromagnetic Calorimeter}
%\subsection{Vertex Detector}
%\subsection{BeamCal}
