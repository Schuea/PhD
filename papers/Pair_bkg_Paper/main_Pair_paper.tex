\pdfoutput=1 % for arXiv to use pdflatex
\documentclass[12pt]{article}

%% The graphicx package provides the includegraphics command.
\usepackage{graphicx}
%% The amssymb package provides various useful mathematical symbols
\usepackage{amsmath}
\usepackage{amssymb}
%% The lineno packages adds line numbers. Start line numbering with
%% \begin{linenumbers}, end it with \end{linenumbers}. Or switch it on
%% for the whole article with \linenumbers after \end{frontmatter}.
\usepackage{lineno}
\usepackage{url}
\usepackage{xspace,multicol}
\usepackage{siunitx}
\usepackage{subcaption}
\usepackage{color, colortbl}
\usepackage{units}
\usepackage{ragged2e}
\usepackage{array}
\usepackage{tabularx}
\usepackage{authblk}
\usepackage{heppennames}
\usepackage{feynmp}
\usepackage{libertine}
% \usepackage[libertine]{newtxmath}
\usepackage{textgreek}
\DeclareGraphicsRule{*}{mps}{*}{}

\newcommand{\guineapig}{GuineaPig\xspace}
\newcommand{\sid}{SiD\xspace}
\newcommand{\electron}{e\textsuperscript{-}\xspace}
\newcommand{\positron}{e\textsuperscript{+}\xspace}
\newcommand{\micron}{\ensuremath{\murm\mathrm{m}\xspace}}

\newcommand{\murm}{%
  \ifmmode
    \mathchoice
        {\hbox{\normalsize\textmu}}
        {\hbox{\normalsize\textmu}}
        {\hbox{\scriptsize\textmu}}
        {\hbox{\tiny\textmu}}%
  \else
    \textmu
  \fi
}


\usepackage{fixltx2e}


\begin{document}

%% Title, authors and addresses

\title{Dependencies of the \positron\electron pair background on the International Linear Collider and the Silicon Detector design}

\author[1,2]{Anne Sch\"utz}

\affil[1]{\normalsize Karlsruhe Institute of Technology (KIT), Department of Physics, Institute of Experimental Nuclear Physics (IEKP), Wolfgang-Gaede-Str. 1, 76131 Karlsruhe}
\affil[2]{\normalsize Deutsches Elektronen-Synchrotron (DESY), Notkestr. 85, 22607 Hamburg}

\maketitle

%%
%% Start line numbering here if you want
%%
%\linenumbers

\begin{abstract}
%% Text of abstract
At AWLC2017 at SLAC\footnote{\url{https://portal.slac.stanford.edu/sites/conf_public/AWLC17/Pages/default.aspx}}, possible changes to the beam parameters of the ILC250 stage were discussed in order to increase the luminosity.
The changes include the reduction of the horizontal beam emittance, leading to a larger electron positron pair background from increased beam-beam interactions.
The solenoid field of the SiD detector forces the pair background particles onto helix tracks which extend to the inner layers of the detector.
This note presents first of all a study of the density of these particles for four different parameter sets of the ILC250 stage.
Additionally, the thereby caused occupancy of the SiD vertex detector was analyzed in order to see the effect of the increased background on the detector.  
The results suggest that the occupancies of all studied schemes can be accommodated in the design of the vertex detector without any loss in performance.
\end{abstract}


%% main text
\section{Introduction}
\label{sec:introduction}
A draft of an ILC change request has been made, discussing new beam parameters of the ILC250 stage with the goal to increase the luminosity and to see a positive effect on the Higgs production cross-section.
Since the new parameter sets include a reduction of the horizontal beam emittance, an increase in beam-beam interaction is to be expected.
This leads to a higher level of \positron\electron pair background and to an effect on the spatial distribution of these pairs.

In the magnetic solenoid field of the SiD detector, the pairs are deflected and form helix tracks originating from the interaction point (IP), and reaching towards the vertex detector.
The performance of the vertex detector plays a crucial role for the precision of physics studies.
Therefore the aim is to keep the overall ratio of so-called dead cells below a per mill level.
By definition, a detector pixel cell is called ``dead'' when no further hits can be stored in its available buffers. 

The plots shown in this note display the envelopes of the pair helix tracks, as well as the vertex detector occupancy for all parameter sets that are under discussion.
Adapting the naming scheme to the change request, the beam parameter schemes are as follows:
\begin{table}
\caption{Possible beam parameter sets for the ILC250 stage. 
The table only lists the parameters that are to be changed with respect to the original ILC250 parameters given in the Technical Design Report (TDR)~\cite[p. 11]{TDR1}.}
\label{tab:ILC250_sets}
\centering
\begin{tabularx}{0.53\textwidth}{llll}
\hline\hline
\textbf{Set}  & \textbf{$\epsilon_x$ [\murm m]} & \textbf{$\beta_x$ [mm]} & \textbf{$\beta_y$ [mm]}\\
\hline
%%\cline{1-4}
%\hline
TDR & 10 & 13.0 & 0.41\\
 (A) & 5 & 13.0 & 0.41\\
 (B) & 5 & 9.19 & 0.41\\
 (C) & 5 & 9.19 & 0.58 \\
\hline\hline
\end{tabularx}
\end{table}

%\input{Sections/VXD_SiD}
%\section{Conclusions}




\section*{Acknowledgments}
The author would like to thank the SiD Optimization group, as well as Daniel Jeans (KEK) for useful discussions and helpful input to the presented studies.

%\include{Appendix}

%% References with bibTeX database:

\bibliographystyle{unsrt}
%\bibliographystyle{model1-num-names} %Doesn't work well if you also cite webpages (Confluence page)
\bibliography{Misc/bibliography.bib}

\end{document}
