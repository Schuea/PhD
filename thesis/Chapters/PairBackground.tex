\chapter{\positron\electron pair background as the largest background contribution}
\label{PairBkg}

\begin{chapterabstract}
 At the International Linear Collider, the collision of the two lepton beams goes hand in hand with the production of background particles.
 Unlike at hadron colliders, the main background contribution does not arise from QCD processes and underlying events, but rather from the interaction of the colliding beam's electromagnetic fields.
 The created secondary \positron\electron pairs form a significant background, the so-called pair background, for the inner detectors, and therefore need to be studied in great detail.
 This chapter discusses the effect of the ILC beam parameters on the pair background, and the impact of the \positron\electron pairs on the \sid detector.
\end{chapterabstract}

As discussed in Section~\ref{BeamBeam}, the pair background is a high cross section process from beam-beam interactions and the main source of background at the ILC.
The secondary electrons and positrons show a characteristic density distribution, which reach to the inner layers of the \sid vertex and tracking detectors.
The impact on the \sid detector is studied with respect to the timing, the hit distribution, and the arising detector occupancy.
These impacts are, however, affected by the change in the ILC beam parameters for the ILC250 stage, which is another study done for this thesis and is explained throughout the following sections.
The results of these studies contributed towards design choices of the accelerator and the \sid detector.

\section{The background generator GuineaPig}
\label{PairBkg:GuineaPig}
For studying the effects of the pair background, \positron\electron pairs from beam-beam interactions were generated with the Monte Carlo (MC) background event generator \guineapig~\cite{Schulte:1997nga} version 1.4.4. 
When providing the accelerator beam parameters, the pair background events of one bunch crossing are simulated and stored in an ASCII output file named ``pairs.dat''.
The parameters used for generating the pair background for this thesis are given in Appendix~\ref{Appendix:Pairs:GuineaPig}. 
\\Since the ASCII files cannot directly serve as input to a full \geant~\cite{geant_ref,geant_ref2} detector simulation, a conversion tool was written in context of this thesis, and instructions on its usage are available in~\cite{Confluence}. 
The tool converts the ASCII output to one of the following common file formats: stdhep or slcio~\cite{LCIO}.
These file formats are directly applicable with the \geant based simulation tool \slic~\cite{Graf:2006ei}, which simulates interactions of the input particles with matter.
The geometry of the simulated world is described in a lcdd file in a human-readable format.
The flexible geometry description allows the simulation of particle interactions with individual detector geometries.
To this end, the geometry description file ``sidloi3'' of the \sid detector, which was used for the simulation studies in this and in the following chapters, is based on the detector design described in Section~\ref{ILC:SiD} and in~\cite[p. 69 ff]{TDR4}.

\section{Pair background envelopes}
\label{PairBkg:helix}
Analyzing the generated pair background events, it becomes apparent that the \positron\electron pairs have a low transverse momentum.
Figure~\ref{fig:PairBkg:Momentum} shows the distribution of their longitudinal and transverse momentum for the two ILC stages at \SI{250}{\GeV} and \SI{250}{\GeV} center-of-mass energy.
\todo{Describe the momentum distributions}
 \begin{figure}[h]
 \centering
  \begin{subfigure}[b]{0.49\textwidth}
   \centering
    \includegraphics[width=\textwidth]{placeholder.jpg}
   \caption{Longitudinal momentum}
   \end{subfigure}
   \hfill
    \begin{subfigure}[b]{0.49\textwidth}
   \centering
    \includegraphics[width=\textwidth]{placeholder.jpg}
   \caption{Transverse momentum}
   \end{subfigure}
   \caption[Pair background momentum distributions]{Comparison of the pair background momentum distributions for the ILC at \SI{250}{\GeV} and \SI{250}{\GeV} center-of-mass energy, with the longitudinal momentum shown in Figure (a) and the transverse momentum in Figure (b).}
   \label{fig:PairBkg:Momentum}
 \end{figure}
\\Due to their low transverse momentum, the pairs are deflected on helical tracks in the magnetic field of the detector solenoid magnet.
An algorithm was written that takes the four-vectors 

\section{Occupancy studies and buffer depth}
\label{PairBkg:occupancy}
Definition of occupancy and dead cells already in BDS muon chapter!
\begin{itemize}
 \item sidloi3
 \item Occupancy studies for ILC250 parameter sets compared with ILC500 TDR
 \item Occupancy studies for ILC250 parameter sets for different SiD designs (old L*, w/o antiDiD etc) - insert plots
 \item Occupancy studies for ILC250 parameter sets in dependency of phi and z - insert plots
\end{itemize}

\subsection{Hit maps of the SiD subdetectors}
\label{PairBkg:hitmaps}
\todo{Show with projection of hitmaps that there are more hits on edges of VertexBarrel detector (because of helix envelopes)}
%TODO TH1D* histo=Layer_1->ProjectionX("ProjectionX",1,Layer_1->GetNbinsY(),"e")


\section{Hit time distributions}
\label{PairBkg:hittime}

\begin{itemize}
 \item Time distribitution of pairs - redo for ILC250
 \item Plots of particle origins - redo for ILC250
 \item Possible reduction of background through time gates
\end{itemize}

