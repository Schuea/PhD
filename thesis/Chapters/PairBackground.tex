\chapter{\positron\electron pair background as the largest background contribution}
\label{PairBkg}
As discussed in Section~\ref{BeamBeam}, the so-called pair background is a high cross-section process from beam-beam interactions and the main source of background at the ILC.


\section{The background generator GuineaPig}
\label{PairBkg:GuineaPig}

\section{Background envelopes}
\label{PairBkg:helix}
\begin{itemize}
 \item Envelopes for ILC250 parameter sets compared with ILC500 TDR - insert plots
\end{itemize}

\section{Occupancy studies and buffer depth}
\label{PairBkg:occupancy}
Definition of occupancy and dead cells already in BDS muon chapter!
\begin{itemize}
 \item sidloi3
 \item Occupancy studies for ILC250 parameter sets compared with ILC500 TDR
 \item Occupancy studies for ILC250 parameter sets for different SiD designs (old L*, w/o antiDiD etc) - insert plots
 \item Occupancy studies for ILC250 parameter sets in dependency of phi and z - insert plots
\end{itemize}

\subsection{Hit maps of the SiD subdetectors}
\label{PairBkg:hitmaps}
\todo{Show with projection of hitmaps that there are more hits on edges of VertexBarrel detector (because of helix envelopes)}
%TODO TH1D* histo=Layer_1->ProjectionX("ProjectionX",1,Layer_1->GetNbinsY(),"e")


\section{Hit time distributions}
\label{PairBkg:hittime}

\begin{itemize}
 \item Time distribitution of pairs - redo for ILC250
 \item Plots of particle origins - redo for ILC250
 \item Possible reduction of background through time gates
\end{itemize}

