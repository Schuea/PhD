\chapter{Introduction}
\label{Introduction}
Since the 1920's, accelerator physics research has been rapidly advancing, and with it so is high-energy particle physics.
Starting with the first accelerating structures, which could accelerate particles to no more than a few hundred \si{\keV}, the reach to higher and higher center-of-mass energies requires new inventions in the field of particle accelerators and colliders.
Along with this goes the development of particle detectors, which need to adapt to the increasingly demanding environments of particle collisions that are created by the high-performing colliders.
Also other areas can benefit from the fast pace in the research efforts and scientific breakthroughs, such as the development of new sensor technologies for mobile phones, new materials for the aerospace industry, or medical applications like cancer therapy.
High-energy and accelerator physics is therefore pioneering in many different scopes.
\\In the field of particle physics, discoveries of new particles and measurements of their characteristics have given explanations and insights to some of the key questions of humanity about the composition of matter, the fundamental forces that describe all physical interactions, and about the beginning of the universe.
The first experimental evidence of the Higgs boson in 2012~\cite{Higgs,Higgs2} was the discovery of one of the last missing pieces in the current understanding of the fundamental building blocks of the universe.
This was accomplished at the Large Hadron Collider (LHC) at CERN~\cite{LHC_CERN}.
With the LHC being the particle collider with the world's highest collision energy of \SI{14}{\TeV}, the peak of the pursuit to the highest energies is reached for the moment.
This does however not mean that the end of this pursuit is near, future colliders at the energy frontier are proposed for the coming decades.
\\After a synopsis of the current knowledge of fundamental particle physics and accelerator physics in Chapters~\ref{Lepton_Physics} and~\ref{LinearColliderPhysics}, it will be derived that colliders at the energy frontier need a complementary collaborator in order to provide a complete understanding of open issues of particle physics, such as the qualities of the Higgs boson.
The International Linear Collider (ILC) is such a collaborator.
It is a proposed linear electron positron collider designed for high-energy physics experiments, not at the energy frontier but at the precision frontier.
Its collision energy will hence be in the range of \SI{250}{\GeV} to \SI{1}{\TeV}, aiming to measure particle physics processes with unprecedented precision.
Chapter~\ref{ILC} will give a full overview of the ILC accelerator design and its detector experiments, and will motivate its physics goals.
\\In order to achieve these goals, which are based on measuring the qualities of particles and their interactions with the highest precision, the detector environment has to rely on minimal background levels.
This is especially important because of the ILC beam timing structure and the readout architecture of the detectors.
However, the background can be constrained through detailed studies of the background sources and by optimizing the accelerator and the detector layouts.
This will be the topic of Chapters~\ref{PairBkg},~\ref{machine_bkg}, and~\ref{BeamDumps}, which are ordered by beam induced backgrounds, machine backgrounds, and backgrounds from the main beam dumps.
The results presented in these chapters have been used to inform design decisions of the ILC accelerator on different topics.
First, the simulations studies concerning the substantial \positron\electron pair background arising from beam-beam interactions will be discussed.
Since this background is directly dependent on ILC beam parameters, various studies of its dependencies have been conducted.
Chapter~\ref{machine_bkg} then investigates the backgrounds from the interaction of the beam with machine components, and the possibilities to reduce these backgrounds.
Finally, the ILC main beam dumps present another source of detector background.
Due to their current design, several issues have to be addressed, not only because of the arising background but also because of the irradiation of the beam dump surrounding.
\\Analyzing the sources of background in detail therefore provides crucial input for the optimization of the ILC accelerator itself.
Afterwards the detectors have to consider the impact of the background in the design of the detector geometry and its readout architecture, and refine them accordingly.
This will bring the ILC closer to its goal of unprecedented precision measurements.