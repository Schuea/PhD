\chapter{Introduction}
\label{Introduction}
Since the 1920's, accelerator physics research has been rapidly advancing, and with it, so is high-energy particle physics.
Starting with the first accelerating structures, which could accelerate particles to no more than a few hundred \si{\keV}, the reach to higher and higher center-of-mass energies requires new inventions in the field of particle accelerators.
Along with this goes the development of particle detectors, which need to adapt to the increasingly demanding environments of particle collisions that are created by the high-performing colliders.
Also other areas benefit from the fast pace in the research efforts and scientific breakthroughs, such as the development of sensor technologies for mobile phones, materials for the aerospace industry, or medical applications like cancer therapy.
High-energy and accelerator physics are therefore pioneering in many different scopes.

In the field of particle physics, discoveries of new particles and measurements of their characteristics have given explanations and insights to some of the key questions about the composition of matter, the fundamental forces that describe all physical interactions, and about the beginning of the universe.
The first experimental evidence of the Higgs boson in 2012~\cite{Higgs,Higgs2} was the discovery of one of the missing pieces in the current understanding of the fundamental building blocks of the universe.
This was accomplished at the Large Hadron Collider (LHC) at CERN~\cite{LHC_CERN}.
With the LHC being the particle collider with the world's highest nominal collision energy of \SI{14}{\TeV}, the pursuit for the highest energies has reached its peak for the moment.
This does however not mean that the end of this pursuit is near, future colliders at the energy frontier are proposed for the coming decades.

After a synopsis of the current knowledge of fundamental particle physics and accelerator physics in Chapters~\ref{Lepton_Physics} and~\ref{LinearColliderPhysics}, it will be derived that colliders at the energy frontier need a complementary collaborator in order to provide a complete understanding of open issues of particle physics, such as the properties of the Higgs boson.
The International Linear Collider (ILC) is such a collaborator.
It is a proposed linear electron positron collider designed for high-energy physics experiments, not at the energy frontier but at the precision frontier.
Its collision energy will hence be in the range of \SI{250}{\GeV} to \SI{1}{\TeV}, aiming to measure particle physics processes with an unprecedented precision that is orders of magnitude better than the LHC.
Chapter~\ref{ILC} will give an overview of the ILC accelerator design and its detector experiments, and will motivate its physics goals.
An important goal is the measurements of the Higgs boson couplings to elementary particles.
Due to the high precision, the ILC will be able to measure these couplings to $\sim$\,\SI{1}{\percent} accuracy, which is needed to distinguish different physics models.
Herein lies the discovery potential of the ILC.

In order to achieve these goals, precision detector experiments are needed with high tracking resolutions and granular calorimeter designs.
To this end, the detector requirements are strict, and emphasize a low material budget for improving the tracking performance.
The ILC aims for nanometer-sized beams to gain high luminosities and to allow the experiments' vertex detector to have a minimal radius.
This leads to high vertex reconstruction efficiencies, but also to the fact that the detector environment has to rely on minimal background levels.
If the occupancy in the innermost subdetectors is too high, the vertex detector performance declines and the ILC goal of unprecedented precision cannot be achieved.

A complete understanding of the detector background and its impact on the detector performance is therefore crucial, especially because of the ILC beam timing structure and the readout architecture of the detectors.
Chapters~\ref{PairBkg} -~\ref{BeamDumps} present detailed studies of various background sources, and show that the background can be constrained by optimizing the accelerator and the \sid detector layout.
\sid is one of the two proposed detector experiments for the ILC, and the focus of the presented detector occupancy studies.
The chapters are sorted by beam induced backgrounds, machine backgrounds, and backgrounds from the ILC main beam dumps.
All of these backgrounds originate inside the detectors (in the case of the beam induced backgrounds), or in parts of the ILC accelerator close to the detectors.
The results of the studies have been used to inform design decisions of the ILC accelerator on different topics.

First, the simulation studies concerning the substantial \positron\electron pair background arising from beam-beam interactions will be discussed in Chapter~\ref{PairBkg}.
Since this background is directly dependent on ILC running schemes, various studies of its dependencies have been conducted.
A study of the timing of the background hits in the individual \sid subdetectors gave hints on how to reduce the background occupancy further.
\\Chapters~\ref{BDS_Muons} and~\ref{machine_bkg} then investigate the backgrounds from the interaction of the beam with machine components, and the possibilities to reduce these backgrounds.
Chapter~\ref{BDS_Muons} presents a simulation study of the muon background from the ILC Beam Delivery System, whereas for Chapter~\ref{machine_bkg} both direct measurements and simulation studies of machine background levels at the Accelerator Test Facility 2 were done.
\\Finally, as discussed in Chapter~\ref{BeamDumps}, the ILC main beam dumps present another source of detector background, and have been studied in a detailed simulation.
Due to their current design, several issues have to be addressed, not only because of the arising background but also because of the irradiation of the beam dump surroundings.

The results of all studies done for this thesis are then summarized in Chapter~\ref{Results}.
It shows that analyzing the sources of background in detail provides crucial input for the optimization of the ILC accelerator itself.
Recommendations are given for optimizing the accelerator with respect to the beam parameters and possible shielding options for the Final-Focus region, which contains the parts of the ILC close to the detectors.
Afterwards, the detector collaborations have to consider the impact of the background in the design of the detector geometry and its readout architecture, and refine them accordingly.
Also here, recommendations for the \sid detector have been made.
This will bring the ILC closer to its goal of measurements at unprecedented precision.