\chapter{Prospects, requirements and limits for the International Linear Collider}
\label{Results}
Detailed studies of different background sources for the International Linear Collider have been presented in Chapters~\ref{PairBkg},~\ref{machine_bkg}, and~\ref{BeamDumps}.
They cover the \positron\electron pair background from beam-beam interactions, the machine background from the interaction of the beam with the beam line components, and the neutron background from the ILC main beam dumps.
\\All of these background sources have been examined in extensive Monte Carlo simulation studies using various physics event generators, such as \guineapig, \mucarlo, \bdsim, and \fluka.
The impact on the \sid detector from the background particles has then been simulated in the \geant based simulation tool \slic, using the \sid simulation infrastructure.
Additionally, the functionality of a vertical beam halo collimator has been tested through measurements of the machine backgrounds at the Accelerator Test Facility 2.
\\Overall, a broad range of background sources has been studied, which has brought insights of the impact of the accelerator design on the background.
The full detector simulations have then shown the effect of the background particles on the \sid detector performance.
The following sections will briefly recap and contextualize the results of the previous chapters.

\section{Keeping the detector background below the critical acceptance limit}

\section{Impact of the ILC running scheme on the background level}

As chapter~\ref{EffectDetectors} has shown, the background sources of a linear collider are dependent on the certain factors, like the beam orbit and the machine parameters.
These dependencies result in requirements and limits that can be formulated for the International Linear Collider.

\section{Possible accelerator optimizations to constrain the background levels}