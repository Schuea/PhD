\chapter{Background from beam-beam interactions}
\label{BeamBeam}
\section{Physics}
\label{BeamBeam:physics}
\section{Pair background}
\label{BeamBeam:pairs}

\begin{figure}
\begin{subfigure}[b]{0.33\textwidth}
\includegraphics[width=\textwidth]{Figures/Bethe-Heitler.pdf}
\caption{Bethe-Heitler}
\end{subfigure}
\begin{subfigure}[b]{0.33\textwidth}
\includegraphics[width=\textwidth]{Figures/Breit-Wheeler.pdf}
\caption{Breit-Wheeler}
\end{subfigure}
\begin{subfigure}[b]{0.33\textwidth}
\includegraphics[width=\textwidth]{Figures/Landau-Lifshitz.pdf}
\caption{Landau-Lifschitz}
\end{subfigure}
\caption[LO Feynman diagrams of the production of the background pairs.]{The LO Feynman diagrams of the production processes of the background pairs: Bethe-Heitler, Breit-Wheeler and Landau-Lifschitz.}
\label{fig:Feynman:pair_production}
\end{figure}


\subsection{GuineaPig}
\section{Bhabha scattering and $\gamma\gamma\rightarrow$hadrons}
\label{BeamBeam:bhabha_gammagamma}

\begin{figure}
\centering
\begin{subfigure}[b]{0.35\textwidth}
\includegraphics[width=\textwidth]{Figures/bhabha_scattering.pdf}
\caption{Bhabha scattering}
\end{subfigure}
\vspace*{0.2cm}
\begin{subfigure}[b]{0.35\textwidth}
\includegraphics[width=\textwidth]{Figures/gammagamma_hadrons.pdf}
\caption{$\gamma\gamma\rightarrow$hadrons}
\end{subfigure}
\caption[LO Feynman diagrams of bhabha scattering and the $\gamma\gamma\rightarrow$hadrons process.]{The LO Feynman diagrams of the bhabha scattering and the $\gamma\gamma\rightarrow$hadrons process.}
\label{fig:Feynman:bhabha_gammagamma}
\end{figure}


\subsection{Pythia}