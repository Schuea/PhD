\chapter{Conclusion}
\label{Conclusion}
The International Linear Collider will be a linear \positron\electron collider at the precision frontier, and therefore complementary to the LHC.
The physics goals of the different ILC stages include measurements of the properties and the interactions of the Higgs boson and the top quark, as well as dark matter and BSM searches.
The aim for these measurements is to have unprecedented precisions.
In order to achieve such levels of precision, a balance has to be found between accelerator design and detector design optimizations with respect to minimizing the detector background.

This thesis has motivated the need for detailed background studies for the ILC.
To this end, direct measurements of machine background levels at the Accelerator Test Facility 2 have been taken for different machine conditions, in order to validate the functionality of a beam halo collimator for the ILC.
This has been done successfully, and all details on the measurements and according Monte Carlo simulation studies are presented in Chapter~\ref{machine_bkg}.
\\In extensive simulations, further background sources have been studied as well.
Thus Chapter~\ref{PairBkg} describes the beam induced \positron\electron pair background and its dependencies on ILC running schemes.
Looking at different beam parameter sets and ILC center-of-mass energies, the pair background was found to be a significant background source, which needs to be confined by both ILC and detector optimizations.
\\Proposed shielding options to prevent the muon machine background from the Beam Delivery System from reaching the detectors are discussed in Chapter~\ref{BDS_Muons}.
Although even the minimal shielding option shields the muons successfully from the detectors, the additional shielding wall serves as tertiary containment device, which is required due to radiation safety regulations.
\\Finally, Chapter~\ref{BeamDumps} analyzes the ILC main beam dumps, which are based on water vessels.
Dumping the ILC beam causes a high radiation dose of the surroundings restricting the duration of stay severely for the maintenance personnel.
Additionally, it creates neutrons traveling back to the interaction region, affecting the outer subdetectors of the detector experiments with respect to the detector occupancy and causing radiation damage.
\\As an alternative approach, gaseous beam dumps have been suggested.

In the process of these analyses, the impact on the \sid detector has been investigated.
By applying the \sid guideline for an acceptable background limit, the occupancies in the detector have been studied, and recommendations have been made accordingly with respect to limiting the background levels below the critical acceptance limit.
These recommendations have also been tested and have been found to be successful.
The results of the presented studies and the given recommendations are summarized in Chapter~\ref{Results}.
They have already been a valuable input to ILC accelerator and \sid design decisions.

Although all of the presented studies are done for the \sid detector only, the generated simulation data have been made available to the ILC community.
With a detailed understanding of the various background sources, the detector background levels can be reduced even further due to refined optimizations of the accelerator and the detectors.
This will enable the ILC and its experiments to achieve their goals of unprecedented precision measurements.