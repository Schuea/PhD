\chapter{Conclusion}
\label{Conclusion}
The International Linear Collider will be a linear \positron\electron collider at the precision frontier, and complementary to the LHC.
The physics goals of the different ILC stages include measurements of the properties of the Higgs boson, the top quark, and their interactions, as well as dark matter and BSM searches.
The aim for these measurements is to have unprecedented precisions.
In order to achieve such levels of precision, a balance has to be found between accelerator design and detector design optimizations with respect to minimizing the detector background.
\\This thesis has motivated the need for the ILC and detailed background studies.
To this end, direct measurements of machine background levels at the Accelerator Test Facility 2 have been taken for different machine conditions, in order to validate the functionality of a beam halo collimator for the ILC.
With Monte Carlo simulations, further background sources and their dependencies on ILC running schemes have been analyzed.
In the process of these analyses, the impact on the \sid detector has also been investigated.
By applying the \sid guideline for an acceptable background limit, the occupancies in the detector have been studied, and recommendations have been made accordingly with respect to limiting the background levels below the critical acceptance limit set by the \sid group.
The results of the presented studies and the given recommendations have already been valuable input to ILC accelerator and \sid design decisions.
\\Although all of the presented studies are done for the \sid detector only, the generated simulation data have been made available to the ILC community.
With a detailed understanding of the various background sources, the detector background levels can be reduced even further due to refined optimizations of the accelerator and the detectors.
This will enable the ILC and its experiments to achieve their goals of unprecedented precision measurements.