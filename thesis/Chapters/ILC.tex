\chapter{The International Linear Collider}
\label{ILC}
The International Linear Collider (ILC) is a linear \electron \positron accelerator with an center-of-mass energy of \unit{500}{GeV} in the first stage and \unit{1}{TeV} in the second.
The state of the art technologies for both, the detectors and the machine, and the machine design make a clean environment and high precision measurements possible, because of which the ILC will be a unique machine.\\
The following sections will talk about the ILC layout, its possible sites, the two detectors, and finally the physics motivation for such a capable accelerator.
\section{Layout}
\label{ILC:layout}

Figure~\ref{fig:ILC_Layout} shows the schematic layout of the ILC for the \unit{500}{GeV} stage.
\begin{figure}
\centering
\includegraphics[width=0.8\textwidth]{Figures/ILC_layout.png}
\caption[Schematic layout of the ILC]{Schematic layout of the ILC~\cite[p. 9]{TDR1}}
\label{fig:ILC_Layout}
\end{figure}

The electrons originate in the polarized source based on a photocathode DC gun.
When these electrons pass through an undulator, high-energy photons are produced which are then converted to electron-positron pairs.
The polarized positrons from these pairs continue in the positron line which is from that point on equivalent to the electron line.
After passing the \unit{5}{GeV} Damping Rings (DR) with a circumference of \unit{3.2}{km}, the electron and positron beams are compressed into bunches by a two-stage compressor system.
Subsequent to that, the bunches are injected into the main linacs.
The main linacs have a length of \unit{11}{km}, and use superconducting radio frequency (RF) cavities with a frequency of \unit{1.3}{GHz} and an average accelerating gradient of \unit{31.5}{MV/m}.
For the machine upgrade to \unit{1}{TeV}, the linacs can be extended in length.
The Beam Delivery System (BDS), which has an overall length of about \unit{4.4}{km}, transports the bunches from the linacs to the Interaction Point (IP).
In the BDS, the beams are focused to nanometer size by the Final Focus (FF) system.
After that, the beams are then finally brought into collision with a crossing angle of \unit{14}{mrad} at the IP.\cite[p. 9-10]{TDR1}
Table~\ref{tab:ILC_parameters} shows the machine parameters for the baseline design at \unit{500}{GeV} center-of-mass energy, and for the luminosity and the energy upgrade stages.

\begin{table}
\caption{Beam parameters for different phases in the ILC operation scenario (Baseline 500, Luminosity Upgrade, TeV Upgrade)~\cite{SiDBkgNote}}.
\label{tab:ILC_parameters}
\centering
\begin{tabularx}{0.74\textwidth}{ll|rrr}
\hline\hline
& & \textbf{Baseline 500} & \textbf{Lumi Upgrade} & \textbf{TeV Upgrade}\\
\hline
\cline{1-5}
\hline
E$_{CM}$  &(\si{\GeV}) & 500  & 500  & \num{1000} \\
n$_b$ & & \num{1312} & \num{2625} & \num{2450}  \\
$\Delta t_b$ &(\si{\nano\second}) & 554  & 366   & 366 \\
N & & \num{2.0e10}  & \num{2.0e10}  & \num{1.74e10}  \\
q$_b$ &(\si{\nano\coulomb}) & 3.2  & 3.2  &  2.7  \\
$\sigma_x^*$ &(\si{\nano\metre}) & 474  & 474  &  481 \\
$\sigma_y^*$ &(\si{\nano\metre}) & 5.9 &  5.9  &  2.8 \\
$\sigma_z$ &(\si{\milli\metre}) & 0.3  &  0.3  &  0.25 \\
L &(\si{\per\centi\metre\squared\per\second}) & \num{1.8e34} & \num{3.6e34} & \num{3.6e34} \\
\hline\hline
\end{tabularx}
\end{table}

The Interaction Region (IR) houses the two detectors for the ILC, the Silicon Detector (SiD) and the International Large Detector (ILD), which are in a push-pull system.
The detectors and the push-pull system are explained in more detail in Section~\ref{ILC:detectors}.

\section{Possible Site}
\label{ILC:site}
Out of originally 10 potential ILC sites in Japan, the Kitakami mountains in the Tohuku Prefecture was chosen to be the preferred site for the ILC.
This decision was made in July 2013 after a detailed study of all site specific factors, like the geological conditions, the infrastructure, and the impact on the environment and the economy.
As can be seen in Figure~\ref{fig:ILC_Site}, the closest city with about 120,000 citizens would be Ichinoseki.
Morioka and Sendai are the biggest cities around the candidate site, with Tokyo being about \unit{430}{km} away.
Although being in the north of Japan, the travel time from Tokyo is only about three hours, and the proximity to the coast line allows the transportation of construction, machine and detector parts by ship.

\begin{figure}
\centering
\includegraphics[width=0.6\textwidth]{Figures/ILC-site.jpg}
\caption[Possible site for the ILC]{The possible site for the ILC are the Kitakami mountains in the Tohuku Prefecture.\cite{Site}}
\label{fig:ILC_Site}
\end{figure}

\section{SiD and ILD}
\label{ILC:detectors}

In order to preserve the competitive spirit and the ability to cross-check results, the ILC has two detectors despite the fact that it is a linear and not a circular collider.
The so-called push-pull system makes that possible by allowing the detectors to switch position after a certain amount of data-taking time.
The whole detector together with the last quadrupole magnet of the accelerator Final Focus system will be pulled out of the beam line, and the other one will be pushed in.
The whole process is designed to take only a short time, i.e. several hours up to 1-2 days, but involves some challenges especially for the magnets, the cryogenics and the detector and machine alignment.\cite[p. 28-29]{TDR1}
The two detectors, the Silicon Detector (SiD) and the International Large Detector (ILD), will be explained in detail in the following with a bigger focus on the SiD detector since some of the background studies presented in the thesis were made for the SiD only.

\subsection{The Silicon Detector}
The SiD is designed to be a robust, compact detector with the vertex and tracker subdetectors and the electromagnetic calorimeter (ECAL) being based on silicon sensors.
The silicon design in comparison to other designs for the vertex and tracking detectors is more robust regarding the beam background and timing.
With also the highly segmented hadronic calorimeter being inside the solenoid field, particle tracking is possible even in the calorimeters.
With being designed to be compact, the measurements for the full detector are \unit{14}{m} in height and \unit{11}{m} in length.
To compensate the small radius, so that SiD is still hermetic and contains the full particle showers, the magnetic field of the superconducting solenoid magnet is \unit{5}{T}.

Figure~\ref{fig:SiD} shows schematic drawings of the SiD detector and its subdetectors.

\begin{figure}
\centering
\includegraphics[width=0.7\textwidth]{Figures/SiD.png}
\caption[Schematic drawing of the SiD detector]{The SiD detector consists of the vertex and tracking detectors (red), the electromagnetic calorimeter (ECAL) (green), the hadronic calorimeter (HCAL) (purple) and the muon system (blue). All subdetectors except the muon system are inside the solenoid magnet.\cite[p. 31]{TDR1}}
\label{fig:SiD}
\end{figure}
%TODO : Update the picture of the SiD detector -> new design for muon system

Overall the SiD detector is optimized for Particle Flow Algorithms (PFA).
PFA is a reconstruction method that reconstructs each particle of the final state individually and uses the different subdetectors for a specific purpose.
Therefore, charged particles are reconstructed from tracks in the tracker device, the ECAL is used for photons, and the ECAL and HCAL together for other neutral particles. 
%TODO : Find a source for explanation of PFA

Table~\ref{tab:KeyParametersSiD} lists the key parameters and measurements of the SiD subdetector systems.

\begin{table}
\caption{Key parameters of the baseline SiD design. All dimensions are given in cm.\cite{SiDBkgNote}}
\label{tab:KeyParametersSiD}
\centering
\begin{tabularx}{0.81\textwidth}{l|llll}
\hline\hline
SiD Barrel & Technology & Inner radius & Outer radius & z extent\\
\hline
Vertex detector & Silicon pixels & 1.4 & 6.0 & $\pm 6.25$ \\
Tracker & Silicon strips & 21.7 & 122.1 & $\pm 152.2$ \\
ECAL & Silicon pixels-W & 126.5 & 140.9 & $\pm 176.5$ \\
HCAL & RPC-steel & 141.7 & 249.3 & $\pm 301.8$ \\
Solenoid & 5 T SC & 259.1 & 339.2 & $\pm 298.3$ \\
Flux return & Scintillator-steel & 340.2 & 604.2 & $\pm 303.3$ \\
\hline
SiD Endcap & Technology & Inner z & Outer z & Outer radius\\
\hline
Vertex detector & Silicon pixels & 7.3 & 83.4 & 16.6 \\
Tracker & Silicon strips & 77.0 & 164.3 & 125.5 \\
ECAL & Silicon pixel-W & 165.7 & 180.0 & 125.0 \\
HCAL & RPC-steel & 180.5 & 302.8 & 140.2 \\
Flux return & Scintillator/steel & 303.3 & 567.3 & 604.2 \\
LumiCal & Silicon-W & 155.7 & 169.55 &  20.0 \\
BeamCal & Semiconductor-W & 326.5 & 344 & 14.0 \\
\hline\hline
\end{tabularx}
\end{table}

\subsection{The International Large Detector}

Like the SiD detector, ILD is a multi-purpose particle detector that is optimized for PFA.
Its vertex detector is also based on silicon sensors, whereas the tracker is a combination of both, a silicon strip and pixel detector and a time projection chamber (TPC).
Also similar to SiD, the calorimeters are within the solenoid magnet, which is only surrounded by the muon system.
The magnetic field of the superconducting solenoid magnet is \unit{3.5}{T} for the ILD.
Because of having a big gaseous volume, the full detector is bigger than SiD, namely \unit{16}{m} in height and \unit{14}{m} in length.
Figure~\ref{fig:ILD} shows all the subdetectors mentioned above in two schematic drawings of the ILD detector.

\begin{figure}
\centering
\includegraphics[width=0.7\textwidth]{Figures/ILD.png}
\caption[Schematic drawing of the ILD detector]{The ILD detector consists of the vertex detector (pink), the time projection chamber (TPC) (yellow), the electromagnetic calorimeter (ECAL) (blue), the hadronic calorimeter (HCAL) (green) and the muon system (gray). All subdetectors except the muon system are inside the solenoid magnet.\cite[p. 34]{TDR1}}
\label{fig:ILD}
\end{figure}
%TODO : Check if I have to update the picture of the ILD detector

\section{Physics Motivation}
\label{ILC:physicsmotivation}

The ILC can be motivated in different ways, of which two will be presented in this section.
First, the advantages it will have over hadron colliders can be summarized with only four key words.
Second, the ILC will be a Higgs factory which will measure the qualities of the Higgs boson with much higher precision than it had been done so far.
The following paragraphs will go into more detail about these points and will motivate why the ILC is a great machine.

\subsection{Key words}
The physics motivation for the ILC can be summarized with four key words: Cleanliness, Democracy, Calculability and Detail.
In the following, the meaning of these key words will be explained with respect to the great advantages the ILC has in comparison to hadron colliders, like the LHC.
\subsubsection{Cleanliness}
Cleanliness describes the clean events the detectors records.
The environment is clean because of the small background levels and therefore small detector occupancy in comparison to a hadron collider.
The energy range is restricted.
There are no out-of-time pileup and no underlying events.
\subsubsection{Democracy}
Because the elementary coupling e of photons is the same for all quarks and leptons, the \electron \positron annihilation produces pairs of all species, SM and exotics, at similar rates.
Besides that, the ILC will be taking data without triggers, i.e. all events will be recorded.
This is possible because of the Cleanliness.
\subsubsection{Calculability}
The energies involved in the initial states are precisely known, since there are only point-like elementary particles in initial state.
Due to the polarized beams, there are only events with couplings to electroweak interactions.
Additionally, systematic uncertainties due to PDF uncertainties and QCD corrections are omitted.
\subsubsection{Detail}
Because of the clean events and the possibility to record and store all taken data, events can be reconstructed in completeness.
The quark and lepton momenta can therefore be determined by kinematic fits.
Studies of the spin-dependence of the production and decay processes are also possible.

\subsection{ILC as a Higgs factory}