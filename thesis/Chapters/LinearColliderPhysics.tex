\chapter{Accelerator Physics of Linear Colliders}
\label{LinearColliderPhysics}

\begin{chapterabstract}
Since the invention of the first particle accelerator in 1930, many new forms of accelerators were discovered and developed over the years. 
Even though they may differ in shape and form, they all rely on the same principles. 
After a brief introduction of these principles of accelerator physics, there will be a description of the two classes of particle accelerators which are mainly used for high-energy physics nowadays: linear and circular colliders. 
By looking at their advantages and disadvantages, the differences between them will be elaborated.
\end{chapterabstract}
\newline

In 1930, J. D. Cockcroft and E. T. S. Walton constructed the first particle accelerator, in order to probe the nuclei of lithium atoms with protons accelerated to several hundred keV. 
They found that the key in particle acceleration lies in electrostatics.

\section{Principles of particle acceleration}
\label{AcceleratorPhysics}
Charged particles are accelerated inside an electric field. 
In a time independent electric field $\vec{E}$ with potential $U$, the particle with charge $q$ experiences a change in its kinetic energy by passing through this electric field.
\begin{equation}
 \Delta E = q \int \vec{E}d\vec{r} = qU
\end{equation}
The particle's charge expressed in the elementary charge e is simply multiplied by the electric potential in Volts to calculate the gain or loss of the particle's kinetic energy. 
Out of convenience, the unit for energy in particle physics is therefore eV.\\
By this logic a particle is accelerated to higher energies by applying higher and higher electrostatic fields. 
Exactly this was done in the beginning of particle accelerators by increasing the voltage applied to a capacitor and shooting the charged particle through. 
Unfortunately, there are limits to the amount of voltage that can be applied before an electric breakdown.
The solution seems simple: putting several capacitors in a row.
But again, this can not be done with electrostatic capacitors since the field gradient between two different capacitors is directed in the opposite direction.
A particle traveling from one capacitor to the next would then lose kinetic energy again.
A solution is quickly found by using time dependent electric fields.\\
In very simple terms, the key principles of a linear collider are thereby already explained.
Better accelerating structures such as drift chambers and later on superconducting radio frequency (RF) cavities were developed over the years.
The first linear accelerator using normal conducting drift chambers of increasing lengths was built by Rolf Wider\o e at the university of Karlsruhe in Germany.
RF fields are applied to the drift chambers such that the particles are accelerated in between the chambers when passing through the gaps.
Since the particle gains energy by passing through a gap, the chambers need to have increasing length in order to guarantee the particle being accelerated by the same phase of the RF field.
\todo{picture of linac with increasing drift chambers}
Nowadays, particle accelerators all over the world mainly use RF cavities instead of drift tubes, with the possibility of being made of superconducting material since the electric resistance is minimal due to their superconducting nature.
Unlike before the acceleration takes place inside the cavities, and all cavities are of the same length and shape.
To still guarantee the particle to be accelerated by the same RF phase, the RF frequency is varied.
\todo{picture of RF cavities}

Rolf Wider\o e did not only build the first linear accelerator, he even more importantly invented the principles of the betatron, the very first accelerating structure using electromagnetic fields.
Also the cyclotron which was invented in 1928 uses a magnetic field to deflect the charged particles on a radial path, and therefore allow the accelerating structure to be much smaller than a linear accelerator.
Exactly that idea was needed to advance particle acceleration from linear to circular.

\section{Linear colliders in comparison to circular colliders}
\label{Linear-Circular}

Circular acceleration does also have certain challenges.
Since the particle gains energy over time, the radius of its circular path increases if the magnetic field is constant.
The accelerator has to be built accordingly taking the increase in the radius into account, or needs to use magnets with variable field strengths.
Ladder is done in synchrotron machines, a type of circular accelerator combining the principles of all particle accelerators mentioned above: acceleration in RF cavities, variation of the RF frequency, and variable magnetic field strengths.
In this way, the particles are traveling along a stationary orbit, passing through the same magnets and cavities over and over again.
To insure the stability of the orbit, not only bending dipole magnets can be found but also sextupole and octupole magnets which focus the particles and correct orbit fluctuations.
Because of being accelerated by an alternating RF field, the synchrotron beams naturally form beam bunches in the following way:
With the particle momentum having a Gaussian distribution, slow particles, i.e. particles with smaller than the ideal momentum, arrive at the accelerating cavity later than the ideal particle, and faster particles earlier respectively.
Therefore particles will be accelerated by different phases of the RF field: particles arriving later will experience a higher phase and will be accelerated more, particles arriving too early will experience a smaller phase and therefore will be accelerated less.
The particles oscillate about the ideal stable phase, and therefore form beam bunches.
\todo{picture of the RF phases: Beschleunigerphysik slides}%BeschleunigerSS12_04, p.5

Currently, the world's largest circular particle collider is the Large Hadron Collider (LHC) at CERN in Switzerland, a synchrotron machine with a circumference of \SI{27}{\kilo\meter}.
The collision energy of the two colliding proton beams is \SI{13}{\TeV}, and its nominal peak luminosity \lumi is \SI{e34}{\centi\meter^{-2}\second^{-1}}.
The luminosity of a particle collider is proportional to the amount of collisions that can occur, and it is defined as:
\begin{align}
 \mathcal{L}&=\frac{N_1N_2 \cdot n_b \cdot f}{2\pi \cdot \sqrt{\sigma^2_{x,1}+\sigma^2_{x,2}} \sqrt{\sigma^2_{y,1}+\sigma^2_{y,2}}}\\
 \intertext{If the bunch sizes of the opposite beams are the same:}
 &=\frac{N_1N_2 \cdot n_b \cdot f}{4\pi \cdot \sigma_x \sigma_y}
\end{align}
$N_{1,2}$ is the number of particles per bunch, which is usually the same for both beams, so that $N_1=N_2$.
$f$ is the revolution frequency, the number of revolutions a bunch makes per second.
$n_{b}$ is the number of bunches, and $\sigma_{x,y}$ is the beam bunch size in the horizontal and the vertical plane.
Table ?? in Chapter~\ref{ILC} lists these parameters and their values for the LHC in comparison to the International Linear Collider (ILC).
In order to translate the luminosity in an event rate, the luminosity value has to be divided by the cross-section $\sigma_p$ of the physics process that is of interest.
Since the LHC detectors do only measure events from inelastic scattering, the LHC event rate can be calculated by taking only the cross-section for inelastic proton-proton scattering into account, which is measured to be \SI{78}{\milli\barn}~\cite{inelXSection}.
\begin{align}
 \dot{N}&=\mathcal{L}\cdot\sigma_{inelastic}\\
 &=10^{34} \textrm{cm}^{-2}s^{-1} \cdot 78\textrm{\,mb}\\
 &=7.8\times 10^8 s^{-1}
\end{align}
%https://www.lhc-closer.es/taking_a_closer_look_at_lhc/0.cross_section
Per second about 780 million events are occurring at the LHC.
%The reason why large synchrotron rings need pre-accelerators is that the range for changing the magnet's field strength is limited.
The LHC is a so-called ``discovery machine''.
With its high luminosity and high collision energy, the possible physics processes from the hadron collisions cover wide energy ranges.  
New particles, first seen for example as resonance peaks in the measured mass spectra, can be discovered easier due to the large phase-space.
This was the case in 2012 for instance, when the LHC found a peak at an energy of \SI{126}{\GeV}, shortly after recognized as the very first measurement of a Higgs boson.~\cite{Higgs}

Why these so-called ``discovery machines'' are hadron and not lepton colliders, and why they are circular and not linear, is to be explained with the physical qualities of the colliding particles.
Hadrons are by definition composite particles, providing the possibility of covering a large energy range when during collision their partons are interacting.
This so-called deep inelastic scattering is explained in more detail in Chapter~\ref{StandardModel}.
In contrast to that, collision of leptons as elementary particles is the interaction of exactly these leptons at exactly their given energy.
This is one of the reasons why lepton colliders are called ``precision machines''.