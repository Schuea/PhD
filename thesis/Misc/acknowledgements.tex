%%% the acknowledgements
%%% put them on a page on their own, but show them without number in the toc
\chapter*{Acknowledgements}
\markboth{Acknowledgements}{Acknowledgements}
\addcontentsline{toc}{chapter}{Acknowledgments} 

First of all, I would like to express my gratitude to my supervisors G\"unter Quast, Eckhard Elsen, and Marcel Stanitzki for giving me the opportunity to develop my scientific skills, my knowledge, and my own personality.
Thanks to the E-JADE funding, I traveled to conferences and workshops all over the world, and spent several months in Japan.
The experiences I gained will be beneficial for my whole life.
\vspace*{0.2cm}\\
\noindent My mentor Thomas Sch\"orner-Sadenius was a great help during the whole time, both work-related and personally.
\vspace*{0.2cm}\\
\noindent I enjoyed the enthusiastic collaboration with the \sid Optimization group.
During the meetings, the knowledge transfer and guidance of the students was extremely fruitful and exemplary.
At this point, I would like to thank especially Jan Strube for his continuous support.
\vspace*{0.2cm}\\
\noindent For their help with proof reading and giving advice on various topics, I am very grateful to Nicholas Walker, Paul Sch\"utze, and Jan Dreyling-Eschweiler.
\vspace*{0.2cm}\\
\noindent For giving me advice and support for my research projects, I would like to thank several people who assisted me on different steps along the way.
First, many thanks to Lewis Keller and Glen White, with whom I collaborated on the muon background study.
Lewis provided me with the generated MUCARLO events, and answered every question I had.
Thank you very much for that!
\vspace*{0.2cm}\\
\noindent From Alfredo Ferrari, Francesco Cerutti, and Vasilis Vlachoudis I got expertise advice on FLUKA, and how to get the results I needed for my ILC beam dump study.
Thank you for your help and your patience.
\\Thanks also to Benno List, who gave me valuable input on the different aspects of the beam dump issues.
\vspace*{0.2cm}\\
\noindent I will never forget the time in Japan, at ATF2 and the University of Tokyo.
Not only the work experience I gained, but also the friends I met are a great gift.
I am very grateful to the ATF collaboration for giving me the chance to work at this accelerator facility, and to learn a great many things about the operation of particle accelerators.
I thank Nuria Fuster-Martinez for our fruitful and harmonic collaboration during and after the data-taking shifts at ATF2.
\\My deepest gratitude also goes to Sachio Komamiya who welcomed me into his working group at the University of Tokyo, and invited me to the beautiful Hanami party.
\vspace*{0.2cm}\\
\noindent I would like to show my appreciation to Stewart Boogert and Laurie Nevay for their support and collaboration on my BDSIM simulation of ATF2.
\vspace*{0.2cm}\\
\noindent But this is not all I took with me from Japan: 
Thanks to all the Terunuma's Turtles!
\\I will cherish the memories of our adventures in and outside of Japan.
I hope the friends we gained will stick together even beyond the world of research.
\\At this point, I would like to thank again Glen White not only for his support during my time as a Ph.D student but also for being a true friend.
\vspace*{0.2cm}\\
\noindent Finally, my biggest thanks goes to Phillip Hamnett.
I am sure the last years would not have been the same without him.
Thank you deeply for all the support and motivation you gave me every day.
\vspace*{0.2cm}\\
\noindent Meinen herzlichsten Dank geht an meine Eltern und meine Familie, die mich immer unterst\"utzen, bei allen Taten und Entscheidungen.
Am Ende dieses Lebensabschnittes m\"ochte ich ihnen danken f\"ur alles, was sie mir erm\"oglicht haben.
Ohne sie w\"urde ich nun sicherlich nicht diese Doktorarbeit in den H\"anden halten k\"onnen.
Danke euch!