\chapter*{Zusammenfassung}
\addcontentsline{toc}{chapter}{Zusammenfassung} 
Der ``International Linear Collider'' (ILC) ist ein geplanter Linearbeschleuniger f\"ur die Kollisionen von Elektronen und Positronen bei einer Schwerpunktsenergie von \SI{250}{\GeV} in der ersten ILC-Phase.
Mit seinen Forschungszielen steht er in einem erg\"anzenden Zusammenhang mit dem ``Large Hadron Collider'' (LHC) am CERN.
Nach der Entdeckung des Higgs-Bosons am LHC in 2012 ist eines der ILC-Ziele, die Eigenschaften und Wechselwirkungen des Higgs-Bosons und des Top-Quarks mit nie dagewesener Präzision zu messen.
Aber auch die Suche nach Teilchen aus Modellen jenseits des Standardmodells, welche ebenfalls auf dem Programm der verschiedenen ILC-Phasen steht, wird durch die Messpräzision erleichtert.
\\Um solche Präzisionen zu erreichen, m\"ussen sowohl das Design des Beschleunigers als auch das der Detektoren optimiert werden, um Untergrundraten unterhalb einer gewissen Grenze zu halten.
Hierzu wurden einige Studien durchgeführt, die unterschiedliche Untergrundquellen untersuchen und die Raten im \sid Detektor, einer der beiden f\"ur den ILC vorgeschlagenen Detektoren, analysieren.
Diese Studien basieren auf Monte Carlo Generatoren, welche Untergrundereignisse aus verschiedenen Quellen simulieren.
In einer vollen Detektorsimulation werden diese Ereignisse dann in Hinsicht auf die \sid Detektorbelegung beurteilt.
Liegt diese Belegung nahe oder sogar oberhalb der von der \sid-Gruppe festgelegten Akzeptanzgrenze, so wurden Vorschläge unterbreitet und getestet, mit denen die Belegung reduziert werden kann.
\\Zu den untersuchten Untergründen gehören der \positron\electron-Paaruntergrund, der durch die Wechselwirkung der elektromagnetischen Felder der kollidierenden Strahlb\"undel entsteht, der Myonen-Maschinenunter-grund aus der Wechselwirkung des Strahls mit den Beschleunigerinstrumenten, und der Neutronenuntergrund, der von den Strahl-Dumps aus in Richtung der Detektoren gerichtet ist.
Die Abhängigkeit des \positron\electron-Paaruntergrundes von den ILC-Strahlparametern wurde evaluiert f\"ur die in 2017 beantragte Änderung der Parameter f\"ur die erste ILC-Phase.
Die Ergebnisse der auch in der vorliegenden Arbeit präsentierten Studie haben dabei zur Entscheidungsfindung entscheidend beigetragen.
\\F\"ur die genannte Studie der Strahl-Dumps wurden zusätzlich die Dump-Designs im Hinblick auf die Bestrahlungsdosis in der Dump-Halle untersucht.
Neben den Simulationsstudien hat auch eine Datennahme stattgefunden, um die Maschinenuntergrundrate in der Beschleunigeranlage ``Accelerator Test Facility 2'' in Japan in Abhängigkeit von verschiedenen Beschleunigerzust\"anden zu messen.
Das Ziel war hierbei die Funktionalität eines Strahlkollimators zu testen.
\\Bei allen präsentierten Studien werden Vorschläge zur Beschleuniger- und Detektoroptimierung genannt, mit denen die Untergrundraten in \sid reduziert und somit die geplanten Pr\"azisionsmessungen erzielt werden können.