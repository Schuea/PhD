\chapter*{Abstract}
\addcontentsline{toc}{chapter}{Abstract} 
The International Linear Collider (ILC) is a proposed linear electron positron collider with a center-of-mass energy of \SI{250}{\GeV} in its first stage.
After the discovery of the Higgs boson at the Large Hadron Collider (LHC) at CERN in 2012, the physics goals of the ILC include the measurements of the Higgs boson properties and its interactions, but also measurements of the top quark and searches beyond the Standard Model are part of the ILC program in the different ILC stages.
The ILC, however, is in competition with the LHC, but is a complementary collider experiment, since it is aimed at unprecedented precisions rather than at high collision energies.
\\In order to achieve such precisions, both the accelerator design and the detector designs have to be optimized with respect to limiting the detector background below an acceptable limit.
For the evaluation of various background sources, different Monte Carlo event generators have been used to generate background events that were then analyzed in a full detector simulation of the \sid detector.
\sid is one of the proposed detector concepts for the ILC, for which a specific critical acceptance limit for background rates has been set.
Throughout the chapters of this thesis, the acceptance limit has been used to assess the arising background occupancy in \sid.
If the occupancy has been found to be close to or to exceed the limit, possibilities to reduce the background level have been tested and recommendations for design optimizations have been made.
\\The presented background simulation studies contain three major background sources: the \positron\electron pair background from beam-beam interactions, the machine background created by interactions of the beam with the accelerator components, and the neutron background produced in the ILC main beam dumps.
The pair background was studied with respect to its dependency on different ILC running schemes, such as the proposed changes in the beam parameter sets for the ILC stage at \SI{250}{\GeV}.
The results of these studies have been used in 2017 to inform the ILC design decision regarding these beam parameters.
\\In addition to the background study for the main beam dumps, the beam dump designs that are based on water dumps have also been analyzed with respect to the arising irradiation of the surroundings.
Besides the mentioned simulation studies, measurements of the machine background in dependency of certain accelerator conditions have been performed at the Accelerator Test Facility 2 at KEK in Japan.
The goal of these measurements was to validate the functionality of a recently installed vertical beam halo collimator that is planned to be used at the ILC as well.
\\For all the presented topics, recommendations for accelerator and detector design optimizations are given, with which the background level in the \sid detector can be reduced in order to facilitate the aimed-for precision at the ILC.