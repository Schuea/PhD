%% LaTeX2e class for student theses
%% thesis.tex
%% 
%% Karlsruhe Institute of Technology
%% Institute for Program Structures and Data Organization
%% Chair for Software Design and Quality (SDQ)
%%
%% Dr.-Ing. Erik Burger
%% burger@kit.edu
%%
%% Version 1.1, 2014-11-21

%% Available languages: english,ngerman
%% Available modes: draft,final (see README)
\documentclass[english,final]{sdqthesis}

%% ---------------------------------
%% | Information about the thesis  |
%% ---------------------------------

%% Name of the author
\author{M. Sc. Anne Sch\"utz}

%% Title (and possibly subtitle) of the thesis
\title{\LARGE Optimizing the design of\\the Final-Focus region for the\\International Linear Collider\\
\hfill
\\ \large Zur Erlangung des akademischen Grades einer\\
DOKTORIN DER NATURWISSENSCHAFTEN (Dr. rer. nat.)}

%% Type of the thesis 
\thesistype{von der KIT-Fakult\"at f\"ur Physik des\\
Karlsruher Instituts f\"ur Technologie (KIT)\\
angenommene und genehmigte\\\vspace*{0.5cm}DISSERTATION}

%% Change the institute here, ``IPD'' is default
% \myinstitute{Institute for \dots}

%% You can put a logo in the ``logos'' directory and include it here
%% instead of the SDQ logo
\grouplogo{DESY_logo}
%% Alternatively, you can disable the group logo
% \nogrouplogo

%% The reviewers are the professors that grade your thesis
\reviewerone{Prof.~Dr.~G\"unter Quast (KIT)}
\reviewertwo{Prof.~Dr.~Eckhard Elsen (CERN)}

%% The advisors
\advisorone{Dr.~Marcel Stanitzki (DESY)}


%% Please enter the start end end time of your thesis
\defensetime{25. Mai 2018}

\settitle

%% --------------------------------
%% | Settings for word separation |
%% --------------------------------

%% Describe separation hints here.
%% For more details, see 
%% http://en.wikibooks.org/wiki/LaTeX/Text_Formatting#Hyphenation
\hyphenation{
% me-ta-mo-del
}

%% ====================================
%% ||                                ||
%% || Beginning of the main document ||
%% ||                                ||
%% ====================================
%% ====================================
\begin{document}

%% Set PDF metadata
\setpdf

%% Set the title
\maketitle
	
%% The Preamble begins here
\frontmatter

%% LaTeX2e class for student theses: Declaration of independent work
%% sections/declaration.tex
%% 
%% Karlsruhe Institute of Technology
%% Institute for Program Structures and Data Organization
%% Chair for Software Design and Quality (SDQ)
%%
%% Dr.-Ing. Erik Burger
%% burger@kit.edu
%%
%% Version 1.1, 2014-11-21

\thispagestyle{empty}
\null\vfill
\noindent\hbox to \textwidth{\hrulefill} 
\iflanguage{english}{I declare that I have developed and written the enclosed
thesis completely by myself, and have not used sources or means without
declaration in the text.\\
\vspace*{0.5cm}\\
Ich versichere wahrheitsgemäß, die vorliegende Doktorarbeit
selbstständig angefertigt, alle benutzten Hilfsmittel vollständig und genau
angegeben und alles kenntlich gemacht zu haben, was aus Arbeiten anderer
unverändert oder mit Änderungen entnommen wurde.}{}\\
 \\
 
%% ---------------------------------------------
%% | Replace PLACE and DATE with actual values |
%% ---------------------------------------------
\noindent\textbf{Hamburg, \today}
\vspace{1.5cm}
 
\noindent\dotfill\hspace*{8.0cm}\\
\hspace*{1.5cm}(\theauthor) 
\cleardoublepage



\end{document}
